\documentclass[12pt,a4paper]{article}
\usepackage[utf8x]{inputenc}
\usepackage[T1]{fontenc}
\usepackage{mathptmx} % Use Times Font

\usepackage[pdftex]{graphicx} % Required for including pictures
\usepackage[swedish]{babel} % Swedish translations
\usepackage[pdftex,linkcolor=black,pdfborder={0 0 0}]{hyperref} % Format links for pdf
\usepackage{calc} % To reset the counter in the document after title page
\usepackage{enumitem} % Includes lists

\frenchspacing % No double spacing between sentences
\linespread{1.2} % Set linespace
\usepackage[a4paper, lmargin=2cm, rmargin=2cm, tmargin=2cm, bmargin=1.8cm]{geometry} %margins
%\usepackage{parskip}

\usepackage[all]{nowidow} % Tries to remove widows
\usepackage[protrusion=true,expansion=true]{microtype} % Improves typography, load after fontpackage is selected

%-----------------------
\hypersetup{ 	
pdfsubject = {},
pdftitle = {Anteckningar, LC},
pdfauthor = {J Jacob Wikner}
}

\title{Anteckningar, Miljögrupperna LC International, Distrikt 101S}
% \author{J Jacob Wikner}

%-----------------------
% Begin document
%-----------------------
\begin{document}
\maketitle

\section{Anteckningar}

\begin{table}[h]
  \center
  \begin{tabular}{p{4cm}|p{11cm}}
    Datum & 2025-04-02 \\ \hline
    Plats & Skälby loge, Kalmar \\ \hline
    Värd för dagens möte &  LC Borgholm \\ \hline
    Närvarande  &  LC Torsås, LC Borgholm, LC Nybro, LC Kalmar, LC Högsby, LC Färjestaden. \newline
    Elva lejon + besökare från Kalmar kommun (men lejonkoppling via släktband) \\ \hline
    Anteckningar & J~Jacob Wikner, LC Mörbylånga 
   \end{tabular}
\end{table}
  

Carl Magnus Persson, LC Mörbylånga, och Kerstin Ahlbäck, LC Borgholm, hälsar välkomna.


\section{Besök av Kalmar Kommun}

Vi har besök av Sanna Andersson, miljö- och klimatstrateg på Kalmar kommun.
Hon ger en överblick om miljöproblem och Kalmar kommuns miljö- och klimatarbete.
I Kalmar kommun jobbar 3 personer med huvudansvar.

\subsection{Lösa anteckningar från presentationer}
Snabba anteckningar från presentation av Sanna. Materialet distribuerades också av Sanna till clubbarna.
\begin{itemize}
  \item Vad är klimatförändringar, varför förändras klimatet, hur förändras klimatet
    Beskriver tidsskalan och hur kort tid som människan varit på jorden.
  \item Vad påverkar jordens klimat. Vulkanutbrott, solens olika faser, jordens lutning, människans påverkan såsom växthusgaser.
  \item Koldioxidutsläpp:
    Våtmarker, skogsbränder, förbränning av fossila bränslen, djur och risodlingar (metan, lustgas).
    Ozonlager och freonet som förstör.
  \item Stark koppling mellan temperaturen och mängden koldioxid i atmosfären.
Finns inga naturliga förklaringar till hög koldioxidnivå (förutom människans avtryck).
Man kan se spår av händelser (såsom pandemi, krig, oljekris och annat) i rapporterat data av mängden koldioxid i atmosfären.
Paradoxalt nog ser man att ett minskat utsläpp av partiklar bidrar till att temperaturen höjs.

\item Man arbetar med ett antal olika prognoser och scenarion för hur utvecklingen de närmsta decennia kommer bli. Det är svårt att nå Parisavtalets mål på 1.5 graders höjning av medeltemperaturen.

\item Scenario, åren 2071--2100. +1.5 - 5 grader varmare, 8-24 dygn högsommar jfr med 3 idag. Blötare, 5-25% mer nederbörd. Men samtidigt --5 - --25\% lägre markfuktighet under sommaren.

\item Medeltemperaturen har väsentligen höjts konstant sedan 1860.

\item Brist på vatten kommer bli ett problem, men även höjning av havsnivån.
Blekinge-Kalmarkusten ett av 10 riskområden för förhöjda havsnivåer.
280 cm höjning ska kunna hanteras av Kalmar kommun och allmän bebyggelse vid slutet av seklet.
160 cm kan det idag bli i extremfall och det har hänt i Kalmar, så sent som 2017.

\item Hur snabbt måste vi agera? Ju närmre vi kommer brytpunkten, ju snabbare måste man agera?

\item Diskussioner kring ansvar, vad kan vi bidra med och hur stort är Sveriges avtryck i det stora hela?
  
\item ``Kalmar kommuns mål är att bli en av Sveriges bästa kommuner när det gäller ekologisk hållbarhet, klimatomställning och klimatanpassning.''
  Minska utsläpp med 85\% från 1990 till 2030.
  Max 15\% får komma från kompletterande åtgärder (koldioxidinlagring).

\item Klimatomställning eller klimatanpassning?
  \textbf{Hindra och/eller lindra?}
  
\item Bruntlandkommissionen (1997): tredimensionell hållbarhet. Samma möjligheter ska säkerställas för framtida generationer. 

\item Kalmar är på plats 29 av 290 kommuner i hur bra man ligger till på miljöarbetssidan.

\item Kalmariten behöver minska utsläppen från 9 ner till 1 ton per år och person 2050.

\item God vattenstatus 2027, Fossilbränslefri klimatneutral kommun 2030 (förnybar energiproduktion), 60/40 Resvalsfördelning\footnote{60\% hållbara transporter, gång, kollektiv, cykel. 40\% bil.} 2035, Livskvalitet i ett grönt Kalmar (biologisk mångfald), cirkulärt samhälle 2045 (delning, återanvändning, byggnation, mm).

\item Exempel på projekt: Containern TaGe, ta och ge -- återbruk. 1200 saker har bytt ägare. Fritidsbanken (låna sportutrustning). Mikroskog inne i staden. Kustmiljögruppen (våtmarker), KLTs och biogasbussar, ekomuddring i Malmfjärden (EU-projekt)

\item Kalmars utmaningar och styrkor: växa hållbart, ojämlikheter, skydd och beredskap, hög utbildningsnivåer.

\item Klimatanpassningsplan (var blir det problem? Äldreomsorg, bebyggelse, mm). Skyfallsplan (säkerställa samhällsviktig verksamhet). Vattenförsörjningsplan (dricksvattenförsörjning).

\item 3-30-300: Se tre träd från hus. 30\% träd per yta. 300m till grönyta.

  \item Ekopodden. Kalmar kommuns podcast.

  \item Tips på sätt att använda sin konsumentmakt.

\item Kom ihåg! Det är inte för sent.

\item Naturvårdsverket har gett ut en bok: ``En varmare värld'', tredje upplagan.
  
  \verb|https://www.naturvardsverket.se/publikationer/1300/en-varmare-varld|

    
\end{itemize}


\section{Vad händer i clubbarna?}
Vi tar varvet runt och beskriver det som sker i de olika clubbarna.

\subsection{LC Borgholm}
Tipspromenad med miljötema veckan innan världsmiljödagen.

\subsection{LC Nybro}
Försöker påtala ämnet miljö hela tiden. Tala inom clubben på möten och nämna ämnet för att medvetandegöra.
Förslag på att plocka skräp under världsmiljödagen.

\subsection{LC Mörbylånga}
Jacob Wikner beskriver arbetet tillsammans med Skansenskolan och Mörbylånga kommun. Det har varit skräpsamling och det finns ett program inför världsmiljödagen. Vi ska ha en popup-loppis i Karlevi i Påsk. Välkomna!

\subsection{LC Högsby}
Inga planer ännu inför världsmiljödagen. Allmänna tankar kring miljö i möten. Undviker plast i sammanhangen.

\subsection{LC Kalmar}
Undviker plast. Försöker samåka där det går.
Studiebesök Moskogen, KSRR.

\subsection{LC Torsås}
Har inte haft en miljöansvarig per se.
Loppis i Gullabo sedan ett antal år. Återanvändning och cirkularitet i åtanke.

(Lionståget, lok med tre vagnar, som lånas av LC Högsby. Insamling av medel. Transporteras med lastbil mellan de olika orterna.)

\subsection{LC Färjestaden}
Loppis 365 dagar om året.
Finns inget i detalj planerat under året. Loppisen är den stora faktorn.

\section{Allmän information och frågar}

\subsection{Zonmöte}
Calle (LC Mörbylånga) informerar om zonmötet i juni och att det ska utses miljökoordinatorer i clubben.

\subsection{Nästa möte -- miljöträffen}

Till nästa möte: titta på miljömanifestet som vi har skapat tidigare. Tänk på hur vi kan uppdatera det och hålla det levande.

Borgholm och Mörbylånga har varit värdar de senaste åren.

LC Färjestaden tar bollen att höra med sin styrelese och återkommer. Lovar att återkomma senast siste april.

Plan B: Kan LC Emmaboda vara ett alternativ. 

\subsection{Inkommet brev}
Ahlbäck, LC Borgholm, presenterar utdrag från mejl från Elisabeth Magnusson, LC Romeleåsen. Information om miljöarbete, mm. Se bilagor.


\end{document}
